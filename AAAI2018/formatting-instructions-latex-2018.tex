\def\year{2018}\relax
%File: formatting-instruction.tex
\documentclass[letterpaper]{article} %DO NOT CHANGE THIS
\usepackage{aaai18}  %Required
\usepackage{times}  %Required
\usepackage{helvet}  %Required
\usepackage{courier}  %Required
\usepackage{url}  %Required
\usepackage{graphicx}  %Required
\usepackage{amsmath,amsfonts}
\newcommand\norm[1]{\left\lVert#1\right\rVert}
\def \y {\mathbf{y}}
\def \x {\mathbf{x}}
\def \Y {\mathbf{Y}}
\def \X {\mathbf{X}}
\def \w {\mathbf{w}}
\def \v {\mathbf{v}}
\def \V {\mathbf{V}}
\def \Z {\mathbf{Z}}
\def \A {\mathbf{A}}
\def \z {\mathbf{z}}
\def \P {\mathbf{P}}
\frenchspacing  %Required
\setlength{\pdfpagewidth}{8.5in}  %Required
\setlength{\pdfpageheight}{11in}  %Required
%PDF Info Is Required:
  \pdfinfo{ 
/Title (Robust and Effective Factorization Machines)
/Author (Anonymous)}
\setcounter{secnumdepth}{0}  
 \begin{document}
% The file aaai.sty is the style file for AAAI Press 
% proceedings, working notes, and technical reports.
%
\title{Robust and Effective Factorization Machines}
\author{Anonymous
}
\maketitle

\section{Optimization Algorithm}
The original objective for classsification task takes the form:
\begin{align}
	\min_{\w \in \mathbf{R}^d,\Z \in \mathbf{S}_{+}^{d \times d}} &\sum_{i=1}^{n}e_i(\max\{\max(y_i(\w^\top\x_i+\langle\Z, \x_i\x_i^\top\rangle),0)-\epsilon_1,0\})^2 \nonumber \\ 
	&+ \frac{\alpha}{2}\norm{\w}^2 + \sum_s\min\{\lambda_s^2, \epsilon_3\},
\end{align}
\begin{align}
e_i=
	\begin{cases}
	\frac{1}{2error},& 0 < error \leq \epsilon_2;\\
	0, & otherwise
	\end{cases}\nonumber
\end{align}
where $error=\max(y_i(\w^\top\x_i+\langle\Z, \x_i\x_i^\top\rangle),0)-\epsilon_1$\\
The subgradient with respect to $\Z$ is
\begin{align}
\bigtriangledown_{\Z, I} = \sum_{i=1}^{b}{\x_i\x_i^\top} + \beta\P_M\P_M^\top\Z 
\end{align}
To incrementally calculate the SVD of $\Z - \eta \bigtriangledown_{\Z,I}$. Let the symmetric and low rank matrix $\Z$ has rank $k$ and its economy SVD is $\Z=\P_k\Sigma_k\P_k^\top$.
As matrix $\bigtriangledown_{\Z,I}$ is symmetric and low rank, we can represent it as $\A \A^\top$.
\begin{align}
\bigtriangledown_{\Z,I}&=\X\X^\top + \beta\P_M\P_M^\top\Z \\ \nonumber
&= \X\X^\top + \beta
\end{align}

\section{Experimental Results}
\begin{table}[htb]
	\centering
		\begin{tabular} {|l|c|c|c|c|}
			\hline
			Dataset & \#Training & \#Test & \#Feature   & \#class \\
			\hline
			Magic04 	& 12680 	& 6340 	 	& 10	& 2\\
			w8a			& 49749		& 14951		& 300	& 2\\
			IJCNN       & 49990     & 91701     & 22	& 2\\
			Covtype		& 387342	& 193670	& 54	& 2\\
			MNIST		& 60000		& 10000		& 784	& 10\\
			epsilon		& 400000	& 100000	& 2000	& 2\\
			\hline
		\end{tabular}
	\caption{Summary of datasets used in our experiments.}
	\label{tab0}
\end{table}
\end{document}
